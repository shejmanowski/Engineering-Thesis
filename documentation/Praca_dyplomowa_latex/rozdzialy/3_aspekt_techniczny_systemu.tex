\chapter{Aspekt techniczny systemu}
Poniższa część pracy skupia się na aspekcie technicznym tworzonego systemu.
Dzięki przygotowanej w poprzednim rozdziale wizji aplikacji, możliwe jest
skupienie się teraz na konkretnej implementacji. Przedstawione zostaną aspekty
takie jak wybór technologii i architektura systemu. Opisany zostanie również
proces implementacji i testowania.

\section{Technologia implementacji}
W dzisiejszym dynamicznym środowisku technologicznym, tworzenie aplikacji
mobilnych stanowi zadanie wymagające przemyślanego podejścia do wyboru
odpowiednich technologii. Wraz z różnorodnością dostępnych platform, języków
programowania oraz narzędzi deweloperskich, napotykamy szeroki zakres możliwości
i wyzwań podczas decydowania, jak zrealizować wizję aplikacji mobilnej. W miarę
jak branża ewoluuje, zauważalne są rozbieżności pomiędzy różnymi technologiami,
a także ich wpływ na wydajność, dostępność oraz doświadczenia użytkownika. Od
natywnych aplikacji po rozwiązania hybrydowe i progresywne, każda opcja posiada
swoje unikalne cechy, a dokonany wybór wpływa zarówno na proces deweloperski,
jak i finalny produkt.

Główne założenia technologiczne obejmowały:
\begin{itemize}
    \item \textbf{Przenośność} - istotną właściwością tworzenej aplikacji była
    możliwość budowania stworzonego sytemu zarówno na telefony posiadające
    system Android, jak i te, bazujące na iOS. Obecne technologie pozwalają na
    budowanie plików wykonywalnych pod różne systemy z tego samego kodu
    źródłowego. Było to kluczowe wymaganie wybieranej technologi, ponieważ w
    znaczącym stopniu wpływało na czas realizacji, dzięki czemu możliwe było
    skupienie się na samej implementacji funkcjonalności, bez zagłębiania się w
    szczegóły przenoszenia kodu między systemami operacyjnymi.
    \item \textbf{Aktualność} - kolejnym ważnym czynnikiem było to, czy dana
    technologia używana jest obecnie w braży. Założeniem było zdobywanie wiedzy
    oraz poznanie mechanizmów wysokopoziomowych technologii, które obecnie
    najczęściej są wykorzystywane.
    \item \textbf{Dokumentacja} - z powodu zagłębiania się w nowe obszary, warta
    uwagi była dokumentacja i stopień jej rozwoju. Poznając nowe języki
    programowania, biblioteki lub narzędzia, często ważniejsze od nich samych,
    jest sposób w jaki zostały opisane. Porządna dokumentacja znacząco zmniejsza
    próg wejścia oraz skraca czas potrzebny do nauki. Dodatkowym aspektem z nią
    związanym jest społeczność danej technologii. W przypadkach, kiedy
    społeczność taka jest duża i aktywna, znacznie łatwiej rozwijać aplikacje.
    Można dzięki niej uzyskać obszerne materiały do nauki, pomoc w konfiguracji
    środowiska lub znaleźć rozwiązanie pewnego problemu implementacyjnego.
    \item \textbf{Skalowalność} - w przypadku chęci późniejszego rozwoju
    aplikacji ważnym czynnikiem jest również łatwość z jaką może się ona
    skalować. W opisywanej implementacji nie był to kluczowy aspekt, jednak było
    to coś wartego uwagi. Nie ponosi się dużych kosztów tworząc od początku
    oprogramowanie, które łatwo będzie skalować. W przeciwnym jednak razie,
    można narazić się na wiele dodatkowej pracy oraz inne problemy w
    przyszłości.
    \item \textbf{Kompatybilność} - działanie każdej aplikacji internetowej
    możemy podzielić na dwie strony. Jedną z nich jest strona prezentacji
    zawartości użytkownikowi - tzw. frontend. Druga zajmuje się
    przetwarzaniem informacji - backend. Najczęściej do realizacji części
    prezentacji po stronie klienta oraz do części logiki po stronie serwera
    wykorzystywane są dwie różne technologie, które muszą się ze sobą sprawnie
    komunikować. W celu płynnego procesu pisania kodu źródłowego, konieczne było
    dobranie odpowiednich technologii w taki sposób aby ich kompatybilność była
    jak najlepsza.
\end{itemize}

\subsection{Frontend}
Narzędzie, który zostało wybrane do tworzenia interfejsu użytkownika (UI) to
Flutter, stworzony przez Google. Bazuje on na obiektowym języku Dart, tej samej
firmy. Jest to nowa technologia, bardzo świeża na rynku - obecna od 2018 roku.
Flutter jest narzędziem wieloplatformowy, co zapewnia, dzięki możliwości
tłumaczenia języka Dart na natywny kod maszynowy dla systemów ARM oraz x86, a
zatem w pełni spełnia wymaganie przenośności tworzonej aplikacji. Jest on już
dziś używany przez wiele dużych firm, a z każdym rokiem ilość ta się zwiększa.
Dokumentacja Fluttera jest niezwykle obszerna i stale rozwijana. Korzystanie z
takich środowisk programistycznych jak Visual Studio Code, umożliwia nam dostęp
do jego dokumentacji po wybraniu dowolnego implementowanego widżetu, bez
konieczności otwierania przeglądarki. Istotnym elementem tego framework'u jest
to, że wszystko co znajduje się we Flutterze to widżet. Kolejne elementy ekranu
to kolejne widżety zagnieżdżone w sobie. Wpływa to na łatwość tworzenia
skomplikowanych ekranów oraz ich reorganizacji. We Fluterze dostępne są dwie
klasy, po których dziedziczy każdy ekran. \textit{StatelessWidget} oraz
\textit{StatefulWidget}. W przypadku, gdy dany ekran ma elementy, które
zmieniają jego stan podczas korzystania z niego (np. licznik, który wyświelta
ile razy kliknęło się w ekran) konieczne jest zastosowanie
\textit{StatefulWidget}, do każdego innego przypadku, kiedy ekran wyświetla
jedynie statyczne informacje, przeznaczony jest \textit{StatelessWidget}. Inne
zalty Fluttera to:
\begin{itemize}
    \item Możliwość szybkiej weryfikacji zmian wprowadzonych w kodzie, dzięki
    funkcji HotReload, która umożliwia wprowadzanie zmian w kodzie przy otwartej
    aplikacji.
    \item Wysoka wydajność, która jest zbliżona do tych oferujących przez
    klasyczne aplikacje natywne.
\end{itemize}

\subsection{Backend}
