\chapter{Studium przypadku zastosowania systemu}
Tak jak wcześniej zostało opisane, w obliczu rosnącej potrzeby dbania o zdrowie 
psychiczne, aplikacje wspierające odporność stają się niezastąpionym 
narzędziem w codziennym funkcjonowaniu. Niniejszy rozdział skupia się na 
kluczowych aspektach dotyczących aplikacji, przeprowadzając szczegółową 
ocenę jej użyteczności poprzez odpowiedzi użytkowników w ankiecie. 

Na wstępie przedstawione zostaną cele projektowanego formularza, jak dostosowano pytania aby 
dowiedzieć się, w jaki sposób aplikacja wpływa na odporność psychiczną
użytkowników oraz czy korzystanie z niej jest intuicyjne. Opisana będzie struktura 
pytań, kryteria doboru uczestników i sposób przeprowadzenia badania.

W ostatniej części rozdziału, omówione zostaną wyniki ankiety i ich 
analiza. Ważne jest, aby na podstawie tak przeprowadzonego kwestionariusza wyciągnąć
dalsze wnioski. Wyszczególnić mocne strony, które posiada aplikacja, co przekonuje 
użytkowników do korzystania z niej. Z drugiej strony istotne jest uświadomienie, jakie 
elementy wymagają poprawy lub co należy dodać, aby zyskać nowych, potencjalnych 
interesariuszy. Te wnioski będą stanowiły fundament dla dalszych działań nad udoskonaleniem
aplikacji oraz dostosowaniem jej do potrzeb użytkowników.

Studium przypadku prezentowane w tym rozdziale ma na celu nie tylko ocenę obecnej
użyteczności aplikacji, lecz również dostarczenie konkretnych wskazówek do
optymalizacji i dostosowywania narzędzia do realnych oczekiwań oraz
doświadczeń użytkowników.

\section{Założenia i przebieg}
\subsection{Cele ankiety}
Podczas tworzenia ankiety dotyczącej działania aplikacji ważne jest, aby
nie skupiać się jedynie na użyteczności, ale także na intuicyjności. Wiele 
aplikacji może być bardzo wartościowych, jednak przez brak łatwego i przejrzystego
interfejsu mogą zniechęcać do korzystania z niej. Takie właśnie cele 
były główną wartoscią podczas doboru pytań. 

Formularz został dostosowany 
pod kątem konkretnych doświadczeń związanych z odpornością psychiczną.
Zadaniem jest zrozumienie, w jaki sposób użytkownicy postrzegają wpływ aplikacji
na ich odporność psychiczną i zdolność radzenia sobie ze stresem, presją,
czy trudnościami życiowymi. Zaobserwowanie subiektywnych doświadczeń interesariuszy
w kontekście konkretnych modułów, funkcji i treści ma posłużyć rozpoznaniu 
obszarów, które są szczególnie efektywne w kształtowaniu odporności psychicznej.

Drugim kluczowym obszarem są pytania dotyczące intuicyjności korzystania z 
aplikacji. Kwestionariusz bada, czy interfejs użytkownika jest postrzegany jako
intuicyjny, czy menu i funkcje są zrozumiałe, a proces korzystania z aplikacji
jest pozbawiony zbędnych trudności. Oceniane będą zdolności użytkowników do
swobodnego poruszania się po aplikacji, korzystania z funkcji oraz szybkiego
odnajdywania poszczególnych elementów. Ma to na celu identyfikację obszarów,
które mogą wymagać poprawek w dziedzinie interakcji użytkownika z systemem.

Zatem ogólnym zadaniem przeprowadzanej ankiety jest uzyskanie kompleksowego 
zrozumienia doświadczeń użytkowników korzystających z aplikacji Good Mentality, 
wpływu systemu na skuteczność wzmacniania odporności psychicznej. Poprzez
połączenie tych dwóch aspektów, możliwe będzie dążenie do stworzenia wizji,
umożliwiającej dostosowanie aplikacji tak, aby nie tylko efektywnie wpływała
na odporność psychiczną, lecz także była łatwa w użyciu i dostępna dla
szerokiej grupy użytkowników. 

\subsection{Struktura pytań}
Pytania zadawane w ankiecie można podzielić na trzy rodzaje, które
zostały opisane w niniejszej części:
\begin{description}
    \item[Ogólne działanie systemu:] Pytania dotyczące ogólnego działania systemu 
    są skoncentrowane na doświadczeniach i odczuciach użytkowników. Próba 
    zobrazowania całokształtu działania aplikacji, czy moduły są zintegrowane
    ze sobą i czy korzystanie przyniosło pozytywne rezultaty, jest niezwykle istotna w kontekście analizy. 
    Przykładowe pytania: "Czy korzystanie z aplikacji wpłynęło na Twoją zdolność radzenia 
    sobie ze stresem? Jeśli tak, opisz pozytywny wpływ.", "Czy działanie aplikaci
    było spójne (czy występowały pojedyncze niespójności w interfejsie lub tematyce)?"
    \item[Intuicyjność:] Drugi obszar, który jest poruszany podczas aniekty dotyczy doświadczeń użytkowników z wytworzonym interfejsem oraz mechaniką aplikacji. Pytania pozwalają uzyskać informacje na temat przemyśleń interesariuszy o łatwości odnajdywania potrzebnych funkcji, trudnościach w nawigacji oraz czy to co widzą jest dla nich przyjazne i zrozumiałe. Przykładowe pytanie:
    "Czy uważasz, że nawigacja między modułami jest łatwa i zrozumiała?"
    \item[Skuteczność poszczególnych modułów:] Ważne jest również stwierdzenie, czy poszczególne elementy aplikacji pozytywanie wpływają na budowanie odporności psychicznej oraz czy treści dydaktyczne umieszczone na stronie okazały się pomocne. Dzięki temu można określić które moduły są szczególnie istotne dla użytkowników, czy pomagają wpłynąć na pozytywny efekt płynący z użytkowania aplikacji. Przykładowe pytanie: "Czy moduł "Sennik" miał istotny wpływ na wspieranie Twojej odporności psychicznej?"
\end{description}
\subsection{Kryteria wyboru ankietowanych}
Aby móc dowiedzieć się potrzebnych rzeczy na temat funkcjonowania aplikacji i jej wpływu na korzystających z niej ludzi, konieczny jest odpowiedni wybór grupy ankietowanych. Umożliwi to uzyskanie kompleksowego obrazu doświadczeń użytkowników związanych z systemem wspierającym wzmacnianie odporności psychicznej. Podczas wyboru takowej grupy wzorowaliśmy się na wymienionych głównych kryteriach:
\begin{description}
    \item[Różnorodność demograficzna:] 
    Aby aniekta miała odpowiednie znaczenie dla badania skuteczności działania, kluczowe jest zapewnienie reprezentatywności pod względem demograficznym. Ankieta obejmuje osoby o zróżnicowanym wieku, płci i poziomie wykształcenia. Różnorodność ta pozwala na uzyskanie globalnej perspektywy na wpływ aplikacji oraz łatwość użytkowania.
    \item[Doświadczenie z podobnymi narzędziami:] Ważne jest również uzyskanie informacji na temat działania systemu od użytowników, którzy pierwszy raz spotkali się z aplikacją tego typu oraz od wcześniej korzystających z podobnych. Różne poziomy doświadczenia ankietowanych z podobnymi aplikacjami wspierającymi zdrowie psychiczne pozwala na zrozumienie różnic w oczekiwaniach między różnymi grupami użytkowników.
    \item[Gotowość do dzielenia doświadczeń:] Kryterium to oejmuje gotowość do dzielenia się wrażeniami i spostrzeżeniami z korzystania aplikacji. Pozbawione sensu jest pokazywanie systemu i pytanie ludzi, którze nie są chętni do podzielenia się swoim zdaniem. Badanie angażuje tylko osoby skłonne przekazać nam swoje uwagi, odczucia, przemyślenia, pozytywne i negatywne aspekty aplikacji.
    \item[Zróżnicowane potrzeby zdrowotne:] Celem ankietyzacji jest między innymi poznanie wpływu aplikacji na odporność psychiczną użytkowników. Wielu cennych wiadomości pozwoli uzyskać przepytanie osób, które posiadają zróżnicowane potrzeby zdrowotne. Badanie uwzględnia osoby, które są narażone na zwiększony stres, lęki, ale także te nieposiadające problemów, a jedynie chcą działać prewencyjnie. Pozwoli to na zrozumienie, jak aplikacja odpowiada na różne potrzeby użytkowników. 
\end{description}
W przypadku badania aplikacji, która wpływa na ludzką psychikę, wzmacniając jej odporność, zakres tematyczny oraz możliwe różne odbieranie funkcjonalności skłania do zaangażowania odpowiedniej liczby ankietowanych. Ostateczna liczba uczestników nie może być zbyt mała, gdyż nie dałoby to adekwatnych i nieprzekłamanych wyników. W przypadku tego kwestionariusza uzyskanie 40 osób chętnych do podzielenia się swoim zdaniem pozwoliło na uzyskanie odpowiedzi zróżnicowanych pod względem nie tylko demograficznym, ale także potrzeb zdrowotnych i zaawansowania w korzystaniu z z podobnych systemów. Dzięki temu można dokładnie zrozumieć różne perspektywy i bardziej precyzyjnie wysnuć wnioski na temat użyteczności i wpływu aplikacji na odporność psychiczną. Dodatkowo takie zróżnicowanie sprawia, że uzyskane wyniki są bardziej reprezentatywne dla ogółu użytkowników aplikacji.

\subsection{Sposób przeprowadzenia badania}
Po zapoznaniu sie wybranych osób z działaniem systemu i jego funkcjonalnościami, zostali oni poproszeni do wypełnienia wcześniej przygotowanego formularza. Ze względu na to, że użytkownicy mieszkają w różnych częściach Polski, ankieta musiała zostać udostępniona online przez określony czas. Zasada anonimowości została starannie zachowana, co miało zwiększyć uczciwość i otwartość udzielanych odpowiedzi. Dodatkowo, w procesie przeprowadzania badania, skoncentrowaliśmy się na zbieraniu danych w sposób, który minimalizuje zakłócenia wynikające z zewnętrznych czynników, zapewniając tym samym, że zebrane informacje są jak najbardziej obiektywne.



\section{Analiza wyników}
Analiza wyników ankiety stanowi kluczowy etap naszego badania nad poprawnością działania aplikacji wspierającej odporność psychiczną. Wśród wielu odpowiedzi, kryją się cenne rady i wskazówki dotyczące efektywności działania systemu. W niniejszej części przyjrzymy się zgromadzonym informacjom, starając się zidentyfikować pozytywne aspekty oraz obszary wymagające potencjalnych popraw. Analiza obejmuje wczesniej opisane kategorie, które zostałty wybrane. Jako efekt rozważań uzyskano głębokie zrozumienie, w jaki sposób aplikacja wpływa na odporność psychiczną oraz jakie kroki można podjąć w przyszłości w celu udoskonalenia aplikacji. Przejście przez wyniki ankiety umożliwi ukierunkowanie działań, jakie zostaną podjęte, dostosowując system do potrzeb interesariuszy.  Znalezienie optymalnego balansu między identyfikacją mocnych stron, a jednoczesnym rozpoznaniem obszarów do poprawy, jest kluczowe dla dążenia do dostarczania narzędzia skutecznego i zaspokajającego różnorodne potrzeby użytkowników w zakresie zdrowia psychicznego. 

Na podstawie analizy odpowiedzi, wynikających z przeprowadzonego badania, wyróżniono kilka pozytywnych aspektów:

\begin{description}
    \item[Intuicyjność interfejsu i łatwość obsługi:] Odpowiedzi potwierdzają, że stworzony interfejs użytkownika można uznać za intuicyjny. Dodatkowo menu oraz poszczególne funkcjonalności są postrzegane za łatwo dostęne. To kluczowy pozytyw, sugerujący poprawną implementację części frontendowej. Oznacza to, że użytkownicy doświadczają płynnego korzystania z systemu.
    \item[Efektywność w Kształtowaniu Odporności Psychicznej:]Jednoznacznie stwierdzono pozytywny wpływ działania aplikacji na stan psychiczny użytkowników. Oznacza to, że wyszukane treści oraz moduły wydają się być skutecznym narzędziem we wspieraniu odporności psychicznej.
    \item[Wygoda w Korzystaniu z Poszczególnych Funkcji:] Użytkownicy wskazali konkretne funkcje, kóre ich zdaniem są szczególnie skuteczne oraz wygodne w korzystaniu. Wnioski wyciągnięte z analizy odpowiedzi sugerują, że te elementy są nie tylko łatwo dostępne, ale także istotnie przyczyniają się do poprawy ogólnej kondycji psychicznej.
    \item[Pozytywne Doświadczenia Użytkownika:]
    Szereg pozytywnych komentarzy od użytkowników podkreśla ogólne zadowolenie z korzystania z aplikacji. Odpowiedzi sugerują, że narzędzie spełnia oczekiwania oraz dostarcza pozytywnych doświadczeń.
\end{description}

Analizując odpowiedzi ankietowanych zaobserwowano kilka obszarów, które wymagają poprawy lub dodania do istniejącego systemu:
\begin{description}
    \item[Personalizacja Doświadczenia Użytkownika:] Niektórzy użytkownicy wyrazili chęć większej możliwości personalizacji aplikacji.Jako jednen z częściej pojawiających się usprawnień określano opcję zmiany języka. Analizując to można dojść do wniosku, że faktycznie może to prowadzić do lepszego zrozumienia treści oraz umożliwuć aplikacji dotarcie do szerszego grona odniorców.
\item[Feedback i Interakcja:]
    W ankiecie pojawiłó się kilka sugestii dotyczących dodatkowych mechanizmów otrzymywania informacji zwrotnej i interakcji w aplikacji. Niektórzy użytkownicy zauważyli, że takie dodatkowe elementy mogą zwiększyć zaangażowanie i motywację do korzystania z systemu.
\end{description}

Podsumowując wyniki analizy zebranych odpowiedzi w ramach przeprowadzanej ankiety, zostało zidentyfikowanych kilka pozytywnych aspektów związanych z działaniem aplikcji. Interfejs użytkownika został oceniony jako intuicyjny, co potwierdza poprawność projektu częśći frontendowej aplikacji. Menu oraz funkcjonalności są postrzegane jako dostępne i łatwo przyswajalne nawet dla użytkowników z małym doświadczeniem z tego typu aplikacjami. Przekłada się to na płynne korzystanie z systemu i przyswajanie treści. Dobrze dobrane funkcjonalności razem z odpowiednim interfejsem pozwalają użytkownikom na odpowiednie działanie, które ma na celu wspieranie odporności psychicznej.  Pozytywne komentarze podkreślają ogólne zadowolenie z korzystania z aplikacji, świadcząc o pozytywnych doświadczeniach użytkowników. Jednocześnie analiza odpowiedzi ujawniła obszary wymagające poprawy. Wyniki badania dostarczają cennych informacji na temat efektywności i użyteczności aplikacji. Pozytywne doświadczenia użytkowników stanowią fundament, jednak identyfikacja obszarów do poprawy jest równie istotna, mając na celu dalszy rozwój systemu.


